\documentclass[addressstd,a4paper,20pt]{dinbrief}

\usepackage[spanish,activeacute]{babel}
\usepackage[T1]{fontenc}
\usepackage[ansinew]{inputenc}
\usepackage{lmodern} %Type1-font for non-english texts and characters

%% Packages for Graphics & Figures %%%%%%%%%%%%%%%%%%%%%%%%%%
\usepackage{graphicx} %%For loading graphic files

\usepackage{mathtools}
\usepackage{listings}
\usepackage{caption}



\usepackage{listings}
 %Type1-font for non-english texts and characters
\usepackage[scaled]{beramono}
%% Images can be included using \includegraphics{Dateiname}
%% resp. using the dialog in the Insert menu.
%% 
%% The mode "LaTeX => PDF" allows the following formats:
%%   .jpg  .png  .pdf  .mps
%% 
%% The modes "LaTeX => DVI", "LaTeX => PS" und "LaTeX => PS => PDF"
%% allow the following formats:
%%   .eps  .ps  .bmp  .pict  .pntg




%% Address of the Sender %%%%%%%%%%%%%%%%%%%%%%%%%%%%%%%%%%%%




\pagestyle{empty} %%No heads and feets



%%%%%%%%%%%%%%%%%%%%%%%%%%%%%%%%%%%%%%%%%%%%%%%%%%%%%%%%%%%%%
%% DOCUMENT - LETTER(S)
%%%%%%%%%%%%%%%%%%%%%%%%%%%%%%%%%%%%%%%%%%%%%%%%%%%%%%%%%%%%%
\begin{document}
\Large


Instituto Polit\'ecnico Nacional
\begin{lstlisting}
\end{lstlisting}
Escuela Superior de C\'omputo
\begin{lstlisting}
\end{lstlisting}
Ana Paola Nava Vivas
\begin{lstlisting}
\end{lstlisting}
2CM4

\begin{lstlisting}
\end{lstlisting}
\begin{lstlisting}
\end{lstlisting}
\begin{lstlisting}
\end{lstlisting}
\begin{lstlisting}
\end{lstlisting}
Acerca de ''Form and Content in Computer Science, by Marvin Minsky'':
\begin{lstlisting}
\end{lstlisting}
\begin{lstlisting}
\end{lstlisting}
\begin{lstlisting}
\end{lstlisting}

La primera secci\'on del art\'iculo, dice que para desarrollar una teor\'ia hay que saber mucho acerca del tema que se va a tratar. El problema con la teor\'ia de la computaci\'on es que no se sabe mucho acerca de este tema como para ense\~nar la materia de manera abstracta, y en cambio se enfocan en ense\~nar los problemas que se conocen bien y que tienen soluci\'on, esperando as\'i, poder averiguar principios m\'as generales.

Resulta ser que cuando se estudia ciencias de la computaci\'on, se le presta m\'as atenci\'on a los problemas conocidos, a clasificarlos, es decir, al formalismo, a ense\~nar muchas cosas pero no a aplicarlas o a aprender por uno mismo. Se hace un esfuerzo enorme por ense\~nar lo canon que hasta parece que matan la creatividad. Lo mismo sucede con la educaci\'on de los ni\~nos peque\~nos, muchas veces el m\'etodo de ense\~nanza les hace aprender varias cosas que si no saben c\'omo utilizar pueden resultar in\'utiles.

La manera de ense\~nar computaci\'on a veces aborda un problema desde un solo punto de vista, y eso no puede ser, para poder comprender un problema se debe hacer como con la f\'isica, analizar desde varias perspectivas porque as\'i es como realmente se comprende un problema.

El formalismo est\'a bien hasta cierto punto, es decir, est\'a muy bien conocer las bases de una ciencia para poder avanzar en ella, pero demasiado formalismo te puede llevar a una encrucijada cuando no puedas aplicar lo que sabes en un problema porque resulta que lo que sabes est\'a mal y no lo sab\'ias porque no te atreviste a cuestionarlo. Demasiado formalismo limita un poco las posibilidades de pensar m\'as all\'a y se corre el riesgo de casarse con una sola manera de ver las cosas, se corre el riesgo de no avanzar m\''as y estancarse en un sola teor\'ia.

Las teor\'ias hay que ponerlas en duda, no se puede creer fielmente algo porque se considere ley, de hecho, as\'i es como avanza la ciencia, probando que se puede ir m\'as all\'a de lo que hoy se considera verdad.

Cuando se educa a los ni\~nos, se les deber\'ia ense\~nar a pensar, a saber de qu\'e sirve lo que se les est\'a ense\~nando. 










\end{document}
